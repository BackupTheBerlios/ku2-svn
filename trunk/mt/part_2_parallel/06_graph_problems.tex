\section{Grafu paralelizēšanas problēmas}
Grafu algoritmu paralelizēšanas procesā var rasties grūtības, kas izriet no grafu algoritmu
īpatnībām, kas atšķir tos no citiem algoritmu veidiem, piemēram, matricu reizināšanas algoritma.
Grafu algoritmi ir uz datiem sakņotas skaitļošanas piemērs
\cite{GraphTheory,HandbookOfGraphTheory,Zettaflops}. Citiem vārdiem sakot,
tie pārsvarā nodarbojas ar datu jeb grafu virsotņu pārmeklēšanu, tas ir, ar pārējām pa saitēm
jeb lokiem, nevis ar intensīvu skaitļošanu. Piemēram, Deikstras algoritma pamatā ir virsotnes
svara noteikšana un virsotņu iezīmes atjaunināšana, bet tas nenodarbojas ar sarežģītām un
intensīvām matemātiskām operācijām kā, piemēram, matricu reizināšanā vai Furjē transformācijā.

\subsection{Horizonta tuvums}
Tas arī nozīme, ka algoritms lielu laiku patērē datu piekļuvei un atjaunināšanai. Pie tam,
nepieciešamais datu apjoms ir salīdzinoši neliels (piemēram, tikai virsotnes iezīme un
blakusvirsotņu saraksts) un nākamie dati, kam būs vajadzīga piekļuve, ir atkarīgi tikai no
tekošajiem datiem (blakusvirsotņu saraksta), tas ir, piekļuve datiem atmiņā pārsvarā ir haotiska
un nav prognozējama. Nav iespējams iepriekš droši noteikt to datu izvietošanos atmiņā,
kas būs vajadzīgi pēc viena soļa. Citiem vārdiem sakot, grafu algoritmu redzamības
horizonts virsotņu un to datu izvietošanos atmiņā ziņā ir ļoti tuvs, proti, viena
soļa attālumā. Tas ir attēlots \reffig{horizon} attēlā.

\numfig{Virsotņu izvietošanās redzamības horizonts}{horizon}{0.8}{img/horizon}

\subsection{Tīkla aizture}
Secīgajos datoros un paralēlajās sistēmās ar kopēju atmiņu tas nav tik būtisks
jautājums, kaut gan tas samazina kešatmiņas izmantošanas iespējas un iespaido algoritma ātrdarbību.
Taču sistēmās ar izkliedētu atmiņu piekļuve datiem ir ļoti svarīgs aspekts, jo tā ir ļoti
dārga operācija. Ja nepieciešami dati ir pieejami lokāli, proti, atrodas lokalajā
operatīvajā atmiņā, tad piekļuves laiks šiem datiem sastāv no pārbaudes, ka dati
tiešām ir lokāli, un datu nolasīšanas no operatīvās atmiņas \seeeqn{read_time}.
Otrā komponente pati par sevi sastāv no darbībām, kas ir saistītas ar virtuālās
atmiņas organizāciju \cite{OSC}, taču tas nav būtisks apskatāmā kontekstā.

\numeqn{read_time}{t_\mathit{piekļuve} = t_\mathit{pārbaude} + t_\mathit{nolasīšana}}


Ja nepieciešamie dati nav pieejami lokāli, proti, tie atrodas uz cita skaitļošanas
elementa, programmai ir jāpieprasa un jāsaņem šie dati izmantojot ziņojumus. Tā kā algoritms
ir uz datiem sakņots, tas visu laiku ģenerē daudz mazu ziņojumu. Tas nozīme, ka būtisku iespaidu
uz algoritma veiktspēju atstāj tīkla laika aizture (latentums). Laika aizture ir konstants laiks,
kas ir nepieciešamas, lai izveidotu ziņojumu [5; 12]. Šis laiks ir liels salīdzinājumā ar
tīkla caurlaides spēju, citiem vārdiem sakot, ir izdevīgāk sūtīt vienu lielu ziņojumu nekā
vairākus mazus.

\subsection{Grafa sadalīšana}
Otrā problēma ir grafa sadalīšana (izkliedēšana) starp skaitļošanas elementiem. Tās mērķis
ir sadalīt grafu tā, lai minimizētu vajadzību pēc “svešu” virsotņu pieprasījumiem. Ideālajā
gadījumā grafs tiek sadalīts gandrīz pilnīgi neatkarīgos apakšgrafos, kas noved komunikāciju
virstēriņu līdz minimumam. Taču vispārīgā situācijā tas nav viegli izdarāms, jo, lai iegūtu
efektīvu sadalījumu, ir jāņem vērā grafa struktūra: grafa virsotņu topoloģija (kā virsotnes
tiek izvietotas “telpā”, kādas virsotnes ir ciešāk saistītas, utml.) un loku semantika (kāda
ir šī loka nozīme, vai tas norāda uz īpašību, vai tas ir “daļa no” attieksme, vai tas parāda
iespējamo pāreju, utml.). Ja šī informācija nav apriori zināma, tad grafa sadalīšana nevar tikt
efektīvi izdarīta “uzreiz” vienā piegājienā, bet ir nepieciešama papildus grafa analīze, kas
palielina algoritma izpildīšanas laiku. Šajā gadījumā var izrādīties, ka izdevīgāk būtu slikti
sadalīt grafu un pazaudēt vairāk laika komunikācijā nekā ilgi meklēt labu sadalījumu un
nezaudēt neko komunikācijā. Taču, ja ir zināms, ka grafa virsotnes ir punkti divdimensiju
telpā, bet loki ir attālumi starp šiem punktiem, tad grafa sadalīšana pēc ģeogrāfiskās pazīmes
(telpu sadala vairākos apgabalos) var dot labus rezultātus. Cits piemērs, ja grafs ir koks,
tad šajā gadījumā var pielietot skaldi un valdi šablonu (no algoritma projektēšanas algoritma
struktūras etapa) un piešķirt katru koka zaru (tas ir, apakškoku, kas ir neatkarīgs no citiem
apakškokiem) savam skaitļošanas elementam.

\subsection{Sinhronizācija}
Viena pieeja, kā zināmā mērā var izvairīties no liela komunikāciju skaita, kas ir saistītas
ar datu pieprasījumiem, ir lokālu kopiju izmantošana. Tas nozīme, ka process lasa un raksta
vērtības lokālajā kopijā nevis vienmēr sūta pieprasījumus pa tīklu. No tā izriet pēdējais
jautājums, uz ko ir jāatbild: kā sinhronizēt lokālās kopijas. Citiem vārdiem sakot, ir nepieciešams
mehānisms, kas nodrošinātu, lai visu skaitļošanas elementu lokālās grafu kopijas saturi būtu
savā starpā saskaņoti un nerastos sacīkstes stāvokļi (situācijas, kad dati var tikt bojāti
“nepareizas” (neparedzētas) operāciju izpildes kārtības dēļ [13], angliski: race conditions).
Sinhronizācijas realizācija ir ļoti svarīga datu korektumam un sistēmas kopējai ātrdarbībai.
Līdz ar ko ir rūpīgi jāizlemj, ko sinhronizēt (kādas noteiktās virsotnes vai visu grafu), ar
ko sinhronizēt (ar visiem skaitļošanas elementiem vai tikai ar to daļu), cik bieži sinhronizēt
un kā realizēt atkāpšanos, ja notika sinhronizācijas kļūda sacīkstes stāvokļa dēļ (piemēram,
ja divi procesi mēģina vienlaikus atjaunināt grafa virsotni utml.).
