\section{MPI}
\emph{Ziņojumu nodošanas saskarne} jeb \emph{MPI} \engl{Message Passing Interface} pašlaik ir
visplašāk izmantotais standarts paralēlajā skaitļošanā. Tā nav jauna programmēšanas valoda,
tā ir apakšprogrammu jeb funkciju bibliotēka, kas primāri var tikt izmantota C un Fortran 77
valodās rakstītās programmās. Šo standartu izstrādā un uztur atvērts, starptautisks forums,
kas sastāv no industrijas, augstākās izglītības iestāžu un vairāku valsts laboratoriju pārstāvjiem.
Tas reglamentē bibliotēkas lietojumprogrammas saskarni. Taču eksistē vairākas šī standarta
realizācijas, piemēram, MPICH [6] vai MS-MPI [7]. MPI pamatā ir ziņojumu nodošana, kas ir
viena no šobrīd visspēcīgākajām un pašlaik visplašāk pielietojamām paralēlās skaitļošanas
paradigmām. Tas ir paredzēts sistēmām ar izkliedētu atmiņu un tās paveidiem [5].

Šobrīd pēdējā standarta versija ir MPI 2.1. Tā tika pieņemta 2008. gada 4. septembrī [8].

Triviāls MPI programmas darbības cikls var būt šāds [1; 5]:
\begin{numlist}
	\item Lietotājs “pasaka” operētājsistēmai palaist programmu. Tā “ievieto” izpildāmā koda
		kopijas (instances) uz katra skaitļošanas elementa.
	\item Katrs skaitļošanas elements sāk izpildīt savu instanci.
	\item Programma inicializē MPI bibliotēku.
	\item Programma identificē savu rangu (MPI terminoloģijā rangs ir unikāls numurs, kas
		tiek piešķirts katrai instancei izpildīšanas sākumā) un kopējo programmas instanču skaitu.
	\item Programmas instance ar rangu, kas tiek uzskatīts par galveno (piemēram, nulle),
		saņem datus no lietotāja un sadala tos starp citām instancēm.
	\item Parējās instances saņem datus, apstrādā tos un nosūta rezultātus atpakaļ.
	\item Galvenā instance apvieno iegūtos rezultātus un nodod galīgos rezultātus lietotājam.
	\item MPI bibliotēkas darbība tiek pabeigta un programma tiek apstādināta.
\end{numlist}

Kā var redzēt, uz visiem procesoriem tiek palaista viena un tā pati programma, kas savas
iekšienē zarojas atbilstoši ranga vērtībai. Šī MIMD programmēšanas forma bieži tiek saukta
kā: viena programma, vairāki dati \engl{Single Program, Multiple Data, SPMD} [1; 5].

Lai gan MPI standartā ir aprakstītas vairāk nekā 120 funkcijas [1; 8], lai uzrakstītu
strādājošo paralēlo programmu, ir nepieciešamas tikai dažas no tām [5]:
\begin{dotlist}
	\item MPI\_Init() un MPI\_Finalize(), bibliotēkas inicializēšana un pabeigšana,
	\item MPI\_Comm\_rank(), sava ranga iegūšana,
	\item MPI\_Comm\_size(), kopējo programmas instanču skaita iegūšana (maksimālais rangs - 1),
	\item MPI\_Send(), ziņojuma sūtīšana,
	\item MPI\_Recv(), ziņojuma saņemšana.
\end{dotlist}

\subsection{,,Sveiki, Pasaule!{}``}
%hhhh\cite{ParProgMPI}, ddd.
\subsection{Masveida komunikācijas}
\subsection{Datu grupēšana}

