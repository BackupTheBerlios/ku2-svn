\section{Lietošanas gadījums}
Par paralelitātes pielietošanas sfēru šī darba ietvaros es izvēlējos grafus un
grafu algoritmus. Šī tēma ir aktuāla, jo grafs ir ērts datu un citu abstrakciju
atspoguļojumu veids tādās, bet ne tikai, disciplīnās, kā bioloģija un ķīmija
(atspoguļojot molekulu struktūras), loģistika (dažādas transporta problēmas),
matemātīka (optimizācija) un, protams, informācijas tehnoloģijas (tīmekļa struktūra,
sistēmu analīze).

Viens no ievērojamākiem grafu piemēriem informācijas tehnoloģijās
ir Internets, kā fragmenta karte ir redzama \reffig{internet} attēlā. Šis
attēls\footnote{Attēlā ir invertētas krāsas, lai parādītu to uz papīra}
ir iegūts Opte projekta\footnote{http://www.opte.org} ietvaros un atspoguļo Interneta
daļas stāvokli uz 2005. gada janvāra mēnesi.

\numfig{Interneta fragmenta karte}{internet}{1}{img/Internet_map_1024}

\subsection{Grafu algoritmi}
Par grafu algoritmiem es saukšu tādus algoritmus, kas ieejā saņem grafu un noteiktā
veidā to analizē un apstrādā. Analizējot grafu, bieži tiešām vai netiešām ir
nepieciešams apskatīt \emph{katru} grafa loku \inbold{ATSAUCE}. Eksistē divas metodes, kā to var
sasniegt, kas ir grafu algoritmu triviāli, bet joprojam, fundamentāli piemēri:
pārmeklēšanas \emph{dziļumā} un \emph{plašumā}. Tieši uz šiem diviem algoritmiem es
akcentēšos savā darbā. Tie ir pietiekami vienkārši, lai tos aprakstītu un analizētu,
bet tajā pat laikā tie mums demonstrē pilnu klāstu bāzes darbību un īpašību, kas
piemīt grafiem. Tas nozīme, šo algoritmu analīze, paralelizēšana un pielietošana
daudzprocesoru sistēmās ar izkliedētu atmiņu atklās visus tos izaicinājumus un
jautājumus, kas ir vispārīgi raksturīgas grafu paralēlajai apstrādei.

Citi grafu algoritmu piemēri ir Deikstras un Flojda algoritmi īsākā ceļa meklēšanai,
Forda-Falkersona algoritms maksimālās plūsmas noteikšanai, algoritmi, kas noteic,
vai grafā ir Hameltona cikls vai algoritmi, kas atrisina ceļojošā pārdevēja problēmu.

Šādi algoritmi ir viena no tādām uzdevumu klasēm, ko ir vērts meģināt risināt izmantojot
paralēlās skaitļošanas tehnikas. Daudzi grafu algoritmi pieder pie NP-pilnas klases.
Šī klase satur tādas problēmas, kas nevar tikt atrisinātas polinomiālā laikā. Taču, ja
risinājums jau ir atrasts, tad to pareizību var pārbaudīt īsā (proti, polinomiālā)
laikā. Kā piemēru var minēt Hameltona cikla, proti, cikla, kas satur visas grafa
virsotnes tikai vienu reizi, esamības noteikšanu. Pati noteikšana ir laikietilpīgs
process, kur, sliktākā gadījumā, ir jāpārbauda visas iespējamas virsotņu kombinācijas.
Savukārt, ja šāds cikls ir atrasts, tad pārbaudīt kā tas tiešām ir Hameltona cikls
ir triviāli: tam ir jāiekļauj visas grafa virsotnes, katra no kām tiek iekļauta
tikai vienu reizi.

Līdz ar to, algoritmu darbības ātrums var būt pietiekami zems, lai tos varētu
efektīvi pielietot lieliem grafiem. Situāciju saasina fakts, ka vairāku algoritmu
darbības laiks nav lineāri atkarīgs no grafa virsotņu
skaita, turpretim, sliktākajā gadījumā šo algoritmu sarežģītība var sasniegt O(n!).
Piemēram, lai atrisinātu ceļojošā pārdevēja uzdevumu 20 virsotnēm, izmantojot pilnās
pārmeklēšanas metodi, ir vajadzīgas 2432902008176640000 iterācijas (jo eksistē 20!
iespējamas virsotņu kombinācijas, no kurām ir jāatrod viena ar mazāko kopējo ceļu)
jeb 38.6 gadus izmantojot vienu procesoru un pieņemot ka procesors izpilda 2 miljardu
iterāciju sekundē. Taču, ja izskaitļošanā piedalās 200 šādu procesoru, tad problēma
ir atrisināma 200 reizes ātrāk jeb 70.4 dienu laikā, ja uzskatīt, ka paātrinājums
ir lineāri atkarīgs no procesoru skaita.

Paralelitātes pielietojuma ieguvums šajā situācijā ir acīmredzams. Bet neskatoties
uz to, grafu paralēlā apstrāde nav plaši izpētīta. Specializēta literatūra par šo
tēmu, ja tā eksistē, ir gruti pieejama. Galvenie spēlētāju šajā jomā
ir Google ar savu Map-Reduce tehnoloģiju, kas nodarbojas ar ar tīmekļa pārmeklēšanas
un analīzes problēmasm, un \inbold{KĀDŠ ĪRIJAS INSTITŪTS}, kas pielieto grafus un paralēlo
skaitļošanu molekulu analīzē.

\subsection{Pārmeklēšana plašumā}
Pārmeklēšana plašumā \engl{Breadth-First Search, BFS} ir grafa pārmeklēšanas
algoritms, kas sāc savu darbību kādā no brīvi izvēlētajām grafa virsotnēm, ko sauc
par pārmeklēšanas sākumvirsotni jeb saknes virsotni, kur atrod visas šai virsotnei
blakus esošās virsotnes. Kādas virsotnes blakusvirsotņu identificēšanas procesu
sauc par šīs virsotnes \emph{izsvēršanu}. Tad, nakāmajā solī katra no atrastajām
virsotnēm tiek izsvērta. Tā tas atkārtojas, kāmēr algoritms neatrod mērķa virsotni.

\numalg{Pārmeklēšana plašumā}{BSF}{
\Procedure{BFS}{$Graph, source$}
	\State izveidot rindu $OPEN$ \Comment tā ir FIFO rinda
	\State izveidot kopu $CLOSED$
	\State \Call{enqueue}{$OPEN, source$} \Comment pievienojam sākumvirsotni rindai
	\State \Call{add}{$CLOSED, source$} \Comment apzīmējam sākumvirsotnei kā apskatītu
	\While{\NotEqual{$OPEN$}{tukšs}}
		\Assign{$v$}{\Call{dequeue}{$OPEN$}}
		    \Comment ņemam pirmo virsotni no rindas
		\ForAll{$w$ $\in$ \Call{incident}{$v$}}
			\Statex \Comment iterējam pa katru blakusvirsotni $w$
			\If{$w$ $\not\in$ $CLOSED$}
			    \Comment ignorējam jau apskatītas virsotnes
				\State \Call{enqueue}{$OPEN, w$}
				\State \Call{add}{$CLOSED, w$}
			\EndIf
		\EndFor
	\EndWhile
\EndProcedure
}

No algoritma redzes viedokļa, visi virsotnes pēcteči pēc tās izsvēršanas tiek
pievienoti FIFO \engl{First In, First Out} rindai. Tipiskajā realizācijā \seealg{BSF}
virsotnes, kas vēl nav apskatītas tiek ievietotas konteinerī (piemēram, sarakstā)
ar nosaukumu \emph{OPEN}, bet virsotnes, kas tika jau izsvērtas, nonāk konteinerī
\emph{CLOSED}.

In general, the edge examination process can be represented by a search tree T, with each vertex in the tree representing a vertex of the graph G, to be analysed. An initial vertex, say v, of G is chosen at random if there is no better choice suspected. It represents the initial vertex in T, representing the initial, zero level of T. Each of its incident edges are examined in turn and are represented by edges incident with v in T. The new vertices incident with these edges represent the first level of T. Each of the unexamined edges, incident with vertices adjacent to v are examined if this is meaningful. In this way a series of walks are identified in G, each represented by a path in T, starting at v. Search paths in T are terminated when either : (i) there are no further incident edges to be examined, or (ii) further examination would be unuseful. A BSF tree along with its graph is shown in Figure 13.3, given on next page.

\subsection{Pārmeklēšana dziļumā}
Depth-first Search. The DFS approach is used in the algorithms for identifying : planarity spanning trees, isomorphism, fundamental cycles and in many other facts. A search tree T, can once again be used to represent the edge examination process. In BFS, each edge incident with a selected vertex v, is examined before selecting a new adjacent vertex. In contrast to it, in DFS a new adjacent vertex is selected, which is incident with the first edge incident with v, i.e., in BFS we stay at v as long as possible, examining all of its incident edges before selecting, examining only one of its incident edges and replacing v by a new vertex, which is adjacent to v. This may leave edges incident with v which are unexamined. These edges may have to be examined later, if the analysis is not resolved in the meantime, or if it is necessary to examine every edge, what ever be the case.

Add a note hereWe now describe a DFS algorithm for the examination of every edge of a connected graph G = (V, E), with n vertices. During the DFS process let

   1.

      Add a note hereN(v) = i imply that vertex v (∈ V) was the ith vertex to be visited.
   2.

      Add a note hereP, F be a partition of E (to be constructed). (P stands for palm and F for frond)

Add a note hereIf G has n vertices, then DFS terminates having labelled the vertices with

    *

      Add a note here{N (v1), N (v2),…‥ , N(vn)} = {1, 2,…‥ , n},

Add a note herein a one-to-one manner. Every edge of E is given an orientation and thus G is transformed into an oriented graph, say digraph D = (V, A). The set P, when edge orientation is taken into account, constitutes a spanning arborescence, called a palm tree. F comprises the edges of E which are not in P. If uv ∈ A then N(u) > N(v).

Add a note hereThe contribution of DFS is to provide a scientific vertex labeling and a spanning arborescence in addition to any particular property.

Add a note here DFS Algorithm Statement. Let G = (V, E) be a connected graph, v0 be a vertex of V at which the process is to begin, and v be the current one being examined.

Add a note here STEPS

   1.

      Add a note hereSet v to become v0,

      Add a note hereP
      	

      Add a note here φ

      Add a note hereF
      	

      Add a note here φ, and

      Add a note here i
      	

      Add a note here0.
   2.

      Add a note hereSet i to become i + 1, and

      Add a note hereN
      	

      Add a note here vi.
   3.

      Add a note here Attempt to identify unexamined edge of E incident with v.
         1.

            Add a note hereIf no such edge exists go to step (5), otherwise
         2.

            Add a note hereChoose any such edge, say vu and examine it. Orient this edge as the arc (v, u). Go to step (4).
   4.

      Add a note here check whether N(u) is defined.
         1.

            Add a note hereIf N(u) is undefined,
                *

                  Add a note hereset P to become P ≈ {(v, u)}, and
                *

                  Add a note here v to become u.
                *

                  Add a note hereGo to step (2).
         2.

            Add a note hereIf N(u) is defined,
                *

                  Add a note hereset F to become F ≈ {(v, u)}.
                *

                  Add a note hereGo to step (3).
   5.

      Add a note hereCheck whether there exists an examined arc, say (y, v) ∈ P.
         1.

            Add a note hereIf such an arc (t, v), exists,
                *

                  Add a note hereset v to become t
                *

                  Add a note hereGo to step (2).
         2.

            Add a note hereIf such an arc does not exist, terminate.

Add a note hereThe vertex numbering assigned to the vertices of a graph by the DFS algorithm is the opposite of logical numbering.

Add a note hereThe oriented graph is shown in Figure 13.4(a) given a head with the palm tree shown with solid lines and the fronds shown in broken lines. The DFS tree is shown in Figure 13.4(b) ahead.
