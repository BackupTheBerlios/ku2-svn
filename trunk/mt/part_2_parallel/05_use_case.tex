\section{Lietošanas gadījums}
Par paralelitātes pielietošanas sfēru šī darba ietvaros es izvēlējos grafus un
grafu algoritmus. Šī tēma ir aktuāla, jo grafs ir ērts datu un citu abstrakciju
atspoguļojumu veids tādās, bet ne tikai, disciplīnās, kā bioloģija un ķīmija
(atspoguļojot molekulu struktūras), loģistika (dažādas transporta problēmas),
matemātīka (optimizācija) un, protams, informācijas tehnoloģijas (tīmekļa struktūra,
sistēmu analīze).

Viens no ievērojamākiem grafu piemēriem informācijas tehnoloģijās
ir Internets, kā fragmenta karte ir redzama \reffig{internet} attēlā. Šis
attēls\footnote{Attēlā ir invertētas krāsas, lai parādītu to uz papīra}
ir iegūts Opte projekta\footnote{http://www.opte.org} ietvaros un atspoguļo Interneta
daļas stāvokli uz 2005. gada janvāra mēnesi.

\numfig{Interneta fragmenta karte}{internet}{1}{img/Internet_map_1024}

\subsection{Grafu algoritmi}
Par grafu algoritmiem es saukšu tādus algoritmus, kas ieejā saņem grafu un noteiktā veidā
to analizē un apstrādā. Grafu algoritmu triviāli, bet joprojam, fundamentāli piemēri
ir pārmeklēšanas dziļumā un plašumā. Tieši uz šiem diviem algoritmiem es akcentēšos
savā darbā. Tie ir pietiekami vienkārši, lai tos aprakstītu un analizētu, bet tajā
pat laikā tie mums demonstrē pilnu klāstu bāzes darbību un īpašību, kas piemīt grafiem.
Tas nozīme, šo algoritmu analīze, paralelizēšana un pielietošana daudzprocesoru
sistēmās ar izkliedētu atmiņu atklās visus tos izaicinājumus un jautājumus, kas ir
vispārīgi raksturīgas grafu paralēlajai apstrādei.

Citi grafu algoritmu piemēri ir Deikstras un Flojda algoritmi īsākā ceļa meklēšanai,
Forda-Falkersona algoritms maksimālās plūsmas noteikšanai, algoritmi, kas noteic,
vai grafā ir Hameltona cikls vai algoritmi, kas atrisina ceļojošā pārdevēja problēmu.

Šādi algoritmi ir viena no tādām uzdevumu klasēm, ko ir vērts meģināt risināt izmantojot
paralēlās skaitļošanas tehnikas. Daudzi grafu algoritmi pieder pie NP-pilnas klases.
Šī klase satur tādas problēmas, kas nevar tikt atrisinātas polinomiālā laikā. Taču, ja
risinājums jau ir atrasts, tad to pareizību var pārbaudīt īsā (proti, polinomiālā)
laikā. Kā piemēru var minēt Hameltona cikla, proti, cikla, kas satur visas grafa
virsotnes tikai vienu reizi, esamības noteikšanu. Pati noteikšana ir laikietilpīgs
process, kur, sliktākā gadījumā, ir jāpārbauda visas iespējamas virsotņu kombinācijas.
Savukārt, ja šāds cikls ir atrasts, tad pārbaudīt kā tas tiešām ir Hameltona cikls
ir triviāli: tam ir jāiekļauj visas grafa virsotnes, katra no kām tiek iekļauta
tikai vienu reizi.

Līdz ar to, algoritmu darbības ātrums var būt pietiekami zems, lai tos varētu
efektīvi pielietot lieliem grafiem. Situāciju saasina fakts, ka vairāku algoritmu
darbības laiks nav lineāri atkarīgs no grafa virsotņu
skaita, turpretim, sliktākajā gadījumā šo algoritmu sarežģītība var sasniegt O(n!). Piemēram,
lai atrisinātu ceļojošā pārdevēja uzdevumu 20 virsotnēm, izmantojot pilnās pārmeklēšanas metodi,
ir vajadzīgas 2432902008176640000 iterācijas (jo eksistē 20! iespējamas virsotņu kombinācijas,
no kurām ir jāatrod viena ar mazāko kopējo ceļu) jeb 38.6 gadus izmantojot vienu procesoru
un pieņemot ka procesors izpilda 2 miljardu iterāciju sekundē. Taču, ja izskaitļošanā piedalās
200 šādu procesoru, tad problēma ir atrisināma 200 reizes ātrāk jeb 70.4 dienu laikā, ja
uzskatīt, ka paātrinājums ir lineāri atkarīgs no procesoru skaita.

Paralelitātes pielietojuma ieguvums šajā situācijā ir acīmredzams. Bet neskatoties
uz to, grafu paralēlā apstrāde nav plaši izpētīta. Specializēta literatūra par šo
tēmu, ja tā eksistē, ir gruti pieejama. Galvenie spēlētāju šajā jomā
ir Google ar savu Map-Reduce tehnoloģiju, kas nodarbojas ar ar tīmekļa pārmeklēšanas
un analīzes problēmasm, un KĀDŠ ĪRIJAS INSTITŪTS, kas pielieto grafus un paralēlo
skaitļošanu molekulu analīzē.

\subsection{Pārmeklēšana dziļumā}
\subsection{Pārmeklēšana plašumā}
