\section{Laiksakrītības atrašana}
Programmatūras projektērājs var strādāt dažādās problēmsfērās. Un jebkūrā no
šīm sfērām projektēšanas process sakās ar problēmas domēnu \engl{problem domain},
proti, ar tādiem elementiem un jēdzieniem, kas
tiešā veidā ir ar to saistīti un kurus izmanto šīs problēmvides eksperti savā
ikdienas darbā. Šajā līmenīnī runa var būt, piemēram, par atomiem, vēju virzieniem
vai saules stārojumu cikliem nevis par šo parādību atspoguļojumiem datu
struktūrās un procedūrās. Šo otro galu devē par programmas domēnu \engl{program
domain}. Tā kā projektēšanas mērķis ir programmatūra, kas pēc būtības ir
dotā šaura problēmsfēras apgabala projekcija uz matēmatiskiem un datora abstrākcijām,
tad kādā brīdī ir jānotiek pārējai no problēmas domēna un programmas domēnu.
Lai gan var šķiet, ka pārējai uz programmas domēnu ir jānotiek pēc iespējas
ātrāk, projektētājs, kas pārāk ātri aiziet no problēmas domēna var palaist garām
vērtīgas projektēšanas iespējas.

Tas ir īpaši aktuāli paralēlajā programmēšanā. Kā jau vairākkārt bija teikts,
paralēlas programmas mēģina atrisināt
lielākas problēmas mazākā laikā vienlaicīgi atrisinot dažādas šīs problēmas daļas
dažādos skaitļošanas elementos. Tomēr, tas var izdoties, tikai ja problēma satur
izmantojamo laiksakrītību, tas ir, uzdevumus, kas \emph{reāli} var tikt izpildīti paralēli.
Taču, pēc tā brīža, kad problēma jau nonāka programmas domēnā, varētu būt ļoti grūti
saredzēt papildus un, var būt, labākas iespējas izmantot laiksakrītību.

Tas nozīme, ka programmētājam ir jāsak paralēla risinājuma projektēšana ar problēmas
analīzi problēmas domēnā. Šāda analīze ietilpst projektēšānas šablonu valodas
laiksakrītības atrašanas fāzē, un tās mērķis ir atrast izmantojamu laiksakrītību.

Viens no svārīgākiem jautājumiem, uz kuru algoritma projektētājam ir jāatbild
pirms ķerties pie darba ir ,,vai ir vērts to darīt?''. Citiem vārdiem sakot,
vai problēma ir pietiekami apjomīga un tās risinājuma vērtīgums ir pietiekami liels,
lai pielietotu spēku un patērētu laiku, lai atrisinātu to ātrāk? Tieši tāpēc šī
darba ietvaros runa ir par lieliem grafiem, jo grafa ar, piemēram, 20 virsotnēm pārmeklēšana
ir perfekti atrisinājama vienprocesoru sistēmās. Jā tas tā ir, tad nākamais solis ir
pārliecināties, ka problēmas pamata īpašības un tajā iesaistīti datu elementi ir
labi saprotami. Problēmsfērasnumfig laba saprašana ir nepieciešama, jo tā palīdz projektētājam
savlaicīgi identificēt un ņemt vērā šai problēmvidei raksturīgus ierobežojumus,
grutības, zemudens akmeņus, iespējas un citas stipras un vājas puses. Pēdējkārt, projektētājam ir
jāidentificē problēmas daļas, kas prasa lielākas skaitļošānas resursus un koncentrēties
uz šo daļu paralelizēšānu.

Tad, pēc \cite{PatParProg} šabloni šajā fazē var tikt organizētas trijās grupās:
\begin{dotlist}
	\item \inbold{Dekompozīcija.} Divi dekompozīcijas šabloni, uzdevumu dekompozīcija
		un datu dekompozīcija, tiek izmantoti problēmas sadalīšanai daļās, kuras
		var tikt izpildītas vienlaicīgi.
	\item \inbold{Atkarību analīze.} Šī grupa sastāv no trīm šabloniem, kas palīdz grupēt
		uzdevumus un analizēt atkarības starp tiem: uzdevumu grupēšana, uzdevumu
		kartošana un datu koplietošana. Parasti šos šablonus pielieto tieši
		norādītajā secībā, taču realitātē bieži ir nepieciešams pārskatīt iepriekš
		pieņemtus lēmumus un atgriezties uz vienu vai diviem soļiem atpakaļ.
	\item \inbold{Projektējuma novērtēšana.} Pēdējais šablons šajā fāzē ir saistīts
		ar projektētāja analīzi par to, kas jau bija līdz šim izdarīts, pirms
		virzities tālāk uz algoritma struktūras fāzi. Šis šablons ir svarīgs,
		jo lielā gadījumu skaitā gādās, ka labākais projektējums nav atrasts pēc
		pirmā mēģinajuma. Jo ātrāk tiks atklātas algoritma projektējuma nepilnvērtības,
		jo vieglāk to būs izlabot.
\end{dotlist}

Kopīga laiksakrītības atrašanas fāzes shēma tiek parādīta \reffig{finding_concurrency}
attēlā.
\numfig{Laiksakritības atrašanas etaps}{finding_concurrency}{0.9}{img/findingConcurrency}
Pirmie trīs posmi (lietderības novērtēšana, problēmas saprašana un paralelizējamas
daļas identificēšana) ir veltīti problēmvidei kā tādai. Tāpēc nākamajās sadaļās
tiks apskatīta problēmsfēra, proti, grafi un to pārmeklēšana. Pēdējie trīs posmi atbilst
šablonu valodai. Tie detalizētāk tiks apskatīti pēc problēmsfēras apraksta.

