\hyphenation{ser-ve-rus pa-ra-lē-lais}

\section{Paralēlās skaitļošanas arhitektūra}
Eksistē vairākas paralēlās skaitļošanas arhitektūras, tādas kā darba staciju tīkli,
tipisku personālo datoru klasteri, paralēlie lieldatori, simetriskās un asimetriskās
multiprocesoru sistēmas \cite{PatParProg}.

\subsection{Flina taksonomija}
Viens no visbiežāk izmantojamiem veidiem, kā var klasificēt un raksturot paralēlos
datorus, ir Flina taksonomija \engl{Flynn`s taxonomy}. Tā apraksta jebkuru datoru
divās dimensijās: instrukcijas un dati. Katra no šīm dimensijām var pieņemt divas
vērtības: viena vai vairākas. Rezultātā ir izdalīti četri datoru tipi: SISD, SIMD,
MISD, MIMD \seefig{flynns-taxonomy} \cite{PatParProg, IntParComp}:
\numfig{Flina taksonomija}{flynns-taxonomy}{0.7}{img/flynn}
\begin{dotlist}
	\item Viena instrukcija, vieni dati \engl{Single Instruction, Single Data, SISD}.
		Tas ir parastais secīgais dators, kur vienā laika momentā procesors izpilda
		vienu programmas instrukciju un darbojas tikai ar vienu datu plūsmu \seefig{SISD}.
		\numfig{SISD}{SISD}{0.8}{img/SISD}
	\item Viena instrukcija, vairāki dati \engl{Single Instruction, Multiple Data, SIMD}.
		Tas ir paralēlais dators. Tas vienā laika momentā izpilda vienu instrukciju,
		kas vienlaikus apstrādā vairākus datu elementus (tās ir tā sauktās vektora
		operācijas) \seefig{SIMD}. Šī koncepcija vislabāk ir piemērota dažiem specifiskiem uzdevumiem,
		piemēram, attēlu vai digitālu signālu apstrādei, taču tīrā veida tiek
		izmantota reti.
		\numfig{SIMD}{SIMD}{0.8}{img/SIMD}
	\item Vairākas instrukcijas, vieni dati \engl{Multiple Instruction, Single Data, MISD}.
		Tas ir paralēlais dators, kas izpilda vairākas instrukcijas vienā laika
		momentā, apstrādājot vienu un to pašu datu plūsmu \seefig{MISD}. Šāda koncepcija var
		būt izmantota, lai, piemēram, vienlaikus pielietotu dažādus kriptogrāfijas
		algoritmus, lai atšifrētu ieejošu ziņojumu vai signālu. Praksē šādu sistēmu
		eksistē ļoti maz.
		\numfig{MISD}{MISD}{0.8}{img/MISD}
	\item Vairākas instrukcijas, vairāki dati \engl{Multiple Instruction, Multiple Data, MIMD}.
		Šī tipa paralēlie datori izpilda vienlaikus vairākas instrukcijas. Katra
		no šīm instrukcijām neatkarīgi strādā ar savu datu plūsmu. Tas ir populārākais
		paralēlo datoru paveids, kam pieder lielāka daļa modernu paralēlu sistēmu.
\end{dotlist}

\subsection{Daudzprocesoru sistēmu topoloģijas}
Visiem datoriem ar vairākiem procesoriem ir noteikts veids kā šie procesori sadarbojas viens
ar otru. Šie veidi atšķiras ar to, kā procesori tiek ,,loģiski izvietoti``, proti, ar to,
kā un un ar ko procesori var apmainīties ar datiem, tādā veidā veidojot komunikācijas tīkla
topoloģiju \cite{ParProgInC}. Dažās sistēmās starpprocesoru komunikācijas tīkls tiek izmantots, lai piekļūtu
kopējai atmiņai. Citās sistēmās~-- lai sūtītu ziņojumus.

\subsubsection{Koplietojama un komutējama vides}
Procesoru sadarbošanās paralēlajā datorā var notikt izmantojot koplietojamu vai komutējamu
vidi \engl{shared or switched media} \cite{ParProgInC}. \emph{Koplietojāmā vidē} tikai viens
ziņojums var tikt sūtīs vienā laikā brīdī. Tās darbības princips ir šāds: procesors raidā
savu ziņojumu šajā vidē, pārējie procesori ,,klausās`` \emph{katru} ziņojumu un atlasa tikai
tos, kas ir domāti tieši šim konkrētajam procesoram. Ethernet ir plaši pazīstams koplietojāmas
vides piemērs.

Parasti šādās vidēs netiek centralizēti koodrinēts, kādam procesoram un kad ir jāsūta sāvs
ziņojums. Turpretīm, katrs process pats izlēmj, kad viņš var sākt pārsūtījumu. Viņš ,,klausās``
vidi kāmēr nekonstatē, ka tā netiek pašlaik izmantota, tad mēģina veikt pārsūtījumu. Ja
divi (vai vairāki) procesori vienlaicīgi mēģina nosūtīt ziņojumus, tad notiek ziņojumu sadursme,
kas nozīme, ka ziņojumi ir jāpārsūta. Šajā gadījumā katrs procesors gaida nejauši izvēlētu
laiku un atkārto mēģinājumu \cite{ParProgInC}. Ziņojumu sadursmes var dramatīski samazināt
augsti noslogotas koplietojāmas vides veiktspēju.

Savukārt, \emph{komutējama pārraides vide} atbalsta tiešas procesors-procesors komunikācijas.
Katram procesoram ir savs ,,ceļš`` līdz komutatoram. Salīdzinot ar koplietojamu vidi, komutatoram
ir divas svarīgas priekšrocības. Pirmkārt, tas atbalsta vienlaicīgu datu pārraidi starp dažādiem
procesoru pāriem, un, otrkārt, tas ļauj mērogot komunikācijas tīklu, iekļāujot tajā arvien
vairāk procesoru.

Komutatoru tīkls var būt atspoguļots ar grafu, kur virsotnes ir procesori un komutatori,
bet loki~-- komunikācijas ceļi. Katrs procesors ir savienots ar vienu komutatoru. Komutatori
savieno procesorus un/vai citus komutatorus.

\subsection{MIMD detalizēts apraksts}
Flina taksonomijas MIMD kategorija ir ļoti plaša un vispārīga. Līdz ar to tā
parasti tiek sadalīta atbilstoši atmiņas organizācijai sistēmās ar kopēju atmiņu
\engl{shared memory}, ar izkliedētu atmiņu \engl{distributed memory}, hibrīdās
sistēmās un gridos \cite{IntParComp, PatParProg}.

Kopējas atmiņas gadījumā visi procesori strādā vienotā adrešu telpā, tas ir, tie
visi var piekļūt vienai un tai pašai adresei atmiņā jeb vienam un tam pašam datu
elementam. Procesoru sadarbošanās notiek lasot un rakstot kopējos mainīgos.

Viena kopējas atmiņas sistēmu klase tiek saukta par simetriskiem multiprocesoriem
jeb SMP (angliski: symmetric multiprocesors). Šādās sistēmās visi procesori ir
savienoti ar kopēju atmiņu tādā veidā, ka jebkura procesora piekļuve jebkuram
atmiņas elementam notiek ar vienādu ātrumu. SMP sistēmas programmēt ir visvienkāršāk,
jo programmētājam nevajag sadalīt datu struktūras pa procesoriem. Taču palielinoties
procesoru skaitam, palielinās noslodze uz atmiņu, tas ir, procesors-atmiņa kopnes
caurlaides spēja ir nopietns ierobežojošs faktors. Tas ir iemesls kāpēc SMP sistēmas
paliek efektīvas tikai pie neliela procesoru skaita.

Cita svarīga klase ir sistēmas ar nevienmērīgu pieeju atmiņai jeb NUMA \engl{nonuniform
memory access}. Šajās sistēmās procesori tāpat kopīgi lieto atmiņu, taču,
piekļuves laiks šai atmiņai ir atkarīgs no tā, cik fiziski ,,tuvi`` vai “tāli” no
procesora atrodas konkrēts atmiņas elements. Citiem vārdiem sakot, katra atmiņas
daļa ir asociēta ar kādu noteiktu procesoru vai procesoriem. Lai samazinātu atmiņas
nevienmērīguma efektu, katram procesoram ir kešatmiņa un mehānisms, kas nodrošina
tās viendabību (koherenci). Šī pieeja ļauj zināmā mērā izvairīties no šaurās vietas,
kas ir saistītas ar kopnes noslodzi, un palielināt procesoru skaitu sistēmā. Bet
no otras puses, tā komplicē programmēšanu. Ja programmētājs grib uzrakstīt programmu,
kas rādītu maksimālu veiktspēju, viņam ir jārūpējas par datu izvietošanu atmiņā
un kešatmiņas efektiem. Kaut gan, NUMA sistēmu programmēšana neatšķiras no SMP
sistēmu programmēšanas.

Izkliedētas atmiņas gadījumā katram procesoram ir savs atmiņas bloks. Tas nozīme
to, ka katrs procesors darbojas savā izolētā adrešu telpā un nevar tiešā veidā
iejaukties cita procesora atmiņā. Procesori sadarbojas, izmantojot ziņojumu nodošanu
(ziņojumu sūtīšanu un saņemšanu) pa kopni jeb maģistrāli, kas savieno procesorus.

Atkarībā no topoloģijas un procesoru savienošanas veida un tajā izmantojamām
tehnoloģijām, komunikācijas ātrums var mainīties no gandrīz tik pat ātram kā
kopējas atmiņas gadījumā (paralēlajos lieldatoros) līdz salīdzinoši lēnam kā tas
ir, piemēram, personālo datoru klasteros, kas ir savienoti ar Ethernet tīklu.
Programmētājam tiešā veidā ir jāprogrammē visas komunikācijas (ziņojumu nodošanu)
starp procesoriem un jānodarbojas ar datu sadalīšanu jeb izkliedēšanu starp tiem.

Datori ar izkliedētu atmiņu tradicionāli tiek sadalīti divās klasēs: paralēlajos
lieldatoros un klasteros. Paralēlajos lieldatoros procesori un tīkla infrastruktūra
atrodas ciešā sasaistē un veido specializētu struktūru paralēlajai skaitļošanai.
Šīs sistēmas ir ļoti mērogojamas, dažos gadījumos tās ļauj skaitļošanā iesaistīt
vairākus tūkstošus procesorus vienlaicīgi.

Klasteri ir izkliedētas atmiņas sistēmas, kas ir veidotas no tipiskajiem datoriem,
savienojot tos ar standartu tīklu. Līdz ar šo datortīklu tehnoloģiju uzlabošanu,
šādas sistēmas kļūst arvien jaudīgākas. Klasteri ir salīdzinoši lēts veids kā
organizācija var izmantot iespēju pielietot paralēlo skaitļošanu.

Hibrīdās sistēmas ir kopējas un izkliedētas atmiņas sistēmu apvienojums. Vairākas
skaitļošanas apakšsistēmas ar kopējas atmiņas arhitektūru savā starpā tiek savienotas
ar kopni vai tīklu. Šādā veidā tiek veidota viena sistēma ar izkliedētu atmiņu.

Gridi ir sistēmas, kas savā darbībā izmanto izkliedētus, heterogēnus resursus
(skaitļošanas serverus, glabātuvi, servisus), kas ir savienoti ar lokālo (LAN)
un/vai teritoriālo (WAN) tīklu. Bieži savienošana notiek izmantojot Internetu.
Gridi var tikt uzskatīti par klasteru vai hibrīdo MIMD mašīnu speciālo gadījumu.
Taču, atšķirībā no klasteriem, gridos visiem resursiem nav jābūt pārvaldītiem no
vienas vietas, tas ir, resursi pieder vairākām organizācijām, kas tos uztur un
nosaka to izmantošanas politikas. Lai izmantotu šādas sistēmas, ir jāizveido speciāla
starpprogrammatūra, kas pārvalda grida darbību un komunikāciju virstēriņu starp
resursiem.

