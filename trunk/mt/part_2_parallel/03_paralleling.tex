\hyphenation{sa-da-la pie-ņem-tie priekš-ro-cī-bas}

\section{Algoritmu paralelizēšana}
Šī sadaļa ir veltīta paralēlo algoritmu izveidošanas aspektiem. Tā ir
balstīta uz informāciju no literatūras avota \cite{PatParProg} par paralēlās
programmēšanas projektēšanas šabloniem.

Algoritma paralelizēšanas jēga ir pārveidot secīgo algoritmu, kas ir paredzēts
SISD sistēmām, par tādu algoritmu, kas būtu spējīgs izmantot MIMD sistēmu
priekšrocības, lai, piemēram, palielinātu savu ātrdarbību, atrodot rezultātu
mazākajā laikā.

\subsection{Izmantojamā laiksakritība}
Vienīgais ceļš pie paralēlās skaitļošanas prakstiskās pielietošanas ved caur
izmantojamo laiksakrītību \engl{exploitable concurrency}. Lai gan var teikt, ka
laiksakrītības potenciāls eksistē jebkurā skaitļošanas problēmā, ja šī problēma
var tikt sadalīta vairākās apakšproblēmās, kas droši un bez kļūdām var tikt
atrisinātas vienlaicīgi. Pie tam, atrisinot problēmu ir iegūti tie
paši rezultāti, kadi būtu bijuši, ja tā tiktu atrisināta klasiskajā secīgajā
veidā. Taču, lai spētu to reāli pielietot programmatūrā, ir jābūt spējīgam
strukturēt programmas kodu tā, lai tas izteiktu un vēlāk, izpildes laikā,
izmantotu laiksakrītību, proti, ļautu izpildīt apakšproblēmas tiešam paralēli.
Citiem vārdiem sakot, laiksakrītībai ir jābūt izmantojamai.

Lielāka daļa lielu skaitļošanas problēmu satur izmantojamo laiksakrītību.
Programmētājs, balstoties uz šādas problēmas analīzi un tajā
identificētu izmantojamo laiksakrītību, veido paralēlu algoritmu vai pārveido
(paralelizē) secīgu algoritmu un implementē to kādā paralēlās programmēšanas vidē.
Kad tāda programma tiek palaista sistēmā ar vairākiem procesoriem, tiek gaidīts,
ka tai vajadzēs mazāk laika, lai atrastu rezultātu. Turklāt, vairāku procesoru
esamība dod iespēju atrisināt lielāka apjoma problēmu risināšanu nekā tas
ir iespējams uz vienprocesora datora.

Vienkāršam piemēram pieņemsim, ka vajag izskaitļot elementu summu no lielas
datu kopas. Ja ir pieejami vairāki procesori, tad tā vietā, ka saskaitīt
visu elementu vērtības pēc kārtas \seefig{par:sum_seq}, datu kopa var tikt sadalīta apakškopās.
Tad, katrā no šīm kopām vienlaicīgi dažādos procesoros tiek atrasta šīs
apakškopas elementu summa. Beigās, šīs daļējās summas tiek saskaitītas,
lai iegūtu galīgo atbildi \seefig{par:sum_par}. Šāda veida risinājums ļauj iegūt pieprasītu
summu ātrāk (kā ir redzams attēlos: 3 pret 7 laika vienībām). Tāpat, kas arī ir svarīgs,
ja katram procesoram ir sava lokālā atmiņa
(tas ir, sistēmu ar izkliedētu atmiņu gadījumā), tad tie spētu strādāt ar
lielāka izmēra datu kopām.

\numfig{Secīga datu kopas elementu summēšāna}{par:sum_seq}{0.55}{img/arraySummSeq}
\numfig{Paralēla datu kopas elementu summēšana}{par:sum_par}{0.51}{img/arraySummPar}

Šis piemērs parāda paralēlas skaitļošanas un programmēšanas būtību. Mērķis
ir izmantot priekšrocības, kuras dod laiksakritība un vairāku procesoru sistēmas,
lai atrisinātu problēmu ātrāk (pārkapjot viena procesora veikstpējas ierobežojumus)
un/vai aptrisitānu lielāka apjoma problēmu (pārkapjot atmiņas apjoma ierobežojumus).
Programmētāja uzdevums
šaja kontekstā ir identificēt laiksakritību problēmā, strukturēt algoritmu,
kas izmantos atrasto laiksakritību, un realizēt algoritmu programmas veidā
piemērotā paralēlas skaitļošanas vidē. Galu galā, problēma tiek atrisināta,
palaistot izstrādātu programmu paralēlajā sistēmā.

Izskatās pietiekami vienkārši, taču, paralēlā programmēšana rada unikālas
problēmas un prasa no programmētāja vairāk iemaņu un pūļu, lai tiktu no
tām vaļā. Laiksakritīgi uzdevumi bieži ir savstarpēji sasaistīti un
visdažādos veidos atkarīgi viens no otra. Šīs sasaistes un atkarības ir
pareizi jāpārvalda, jo kārtība, kadā uzdevumi tiek izpildīti var nenoteiktā
veidā ietekmēt gala rezultātus. Šādu uzvedību devē par sacīkstes stāvokļiem
\engl{race condition}, un par to plašāk tiks pastastīs talāk. Bet tagad atgriezamies
pie elementu summešanas. Pieņemsim, ka runa ir par darbībām ar reāliem nevis
veseliem skaitliem. Tad, ņēmot vērā, ka dators darbojas ar šādiem skaitļiem tikai
ar noteikto precezitāti \inbold{ATSAUCE}, gala rezultāti var atšķirties atkarībā
no summēšanas kartības.

Pat gadījumā, ja paralēlā programma ir ,,pareiza`` gan no sacīkstes stāvokļu,
gan no citu paralēlās programmēšanas loģikas aspektu viedokļa, tā var neattaisnot
izstrādātāja cērības, piemēram, veiktspējas uzlabošanas ziņā. Īstenībā, programmas
veikstspēja var par pazemināties. Viens no tā iemesliem varētu būt, ka virsteriņš,
ko izraisa laiksakritības pārvaldība, ir parāk liels un ,,apēd`` lielāku daļu
sistēmas resursu. Tāpat, uzdevumu sadalīšana pa procesoriem ne vienmēr ir tik
triviāla kā summēšanas piemērā, jo paralēlā algorita efektivitāte ir atkarīgā arī
no tā, kā tas ,,atbilst`` paralēlajam datoram, uz kā tas tiek palaists. Algoritms
var būt ļoti efektīvs uz vienas datora arhitektūras, bet pilnīgi nelietojams uz citas.

\subsection{Projektēšanas šabloni un šablonu valodas}
Lai atvieglotu darbu programmētājam, kas raksta paralēlās programmas, proti,
veido paralēlos algoritmus, tika izstrādāti paralēlās programmēšanas projektēšanas
šabloni \engl{design patterns for parallel programming}, kas pēc būtības ir
vadlīniju kopums, kas apraksta, kā programmētājam vajag rīkoties noteiktajā uzdevumu
atrisināšanas stādijā noteiktajā kontekstā. Šie šabloni apkopo bieži sastopamas
situācijas un problēmas un ar tām saistītus ekspertu viedokļus un pieredzi.

Bez risinājuma ir svarīgs arī šablona nosaukums jeb vārds. Tas formē dotajai
problēmvidei specifisku vārdnīcu, kas uzlabo komunikāciju starp šīs problēmsfēras
projektētājiem, programmētājiem un citām iesaistītām personām un samazina
pārpratumu varbūtību, kas galu galā vēd pie kvalitatīvāka produkta.

Kā papildinājums projektēšanas šabloniem eksistē šablonu valoda \engl{pattern language}.
Tā organizē šablonus noteiktajā secībā, kā solis pa solim ved valodas lietotāju
caur šabloniem, tādā veidā ļaujot izveidot sarežģītākās sistēmas. Katrā lēmuma
punkta projektētājs izvēlas viņam nepieciešamu šablonu. Katrs šablons, savukārt, ved pie
nākama lēmumu punkta, tas ir, pie citiem šabloniem, un tā rekursīvi talāk.
Rezultātā tiek iegūts algoritmu gala projektējums, kas var tikt identificēts ar
izvēlētu lēmumu kopu un to secību. Šie lēmumi un to secība ir algoritma konfigurācijas
daļa.

\subsection{Šablonu valoda paralēlajai programmēšanai}
Paralēlajai programmēšanai specifiskā šablonu valoda, proti, paralēlā algoritma
projektēšanas process var tikt sadalīts, kā tas tiek parādīts \reffig{par:pat_overview}
attēlā, četros secīgos etapos: laiksakritības
atrašana \engl{finding concurrency}, algoritma struktūru \engl{algorithm
structure}, atbalsta struktūru \engl{supporting structures} un realizācijas
mehānismu \engl{implementation mechanisms} izvēle.

\numfig{Šablonu valodas pārskats}{par:pat_overview}{0.35}{img/patternOverview}

Laiksakritības atrašanas etaps ir saistīts ar problēmas strukturizāciju ar mērķi identificēt
iespējamo laiksakritību. Šajā līmenī projektētājs fokusējas uz augsta līmeņa algoritmiskiem
jautājumiem un pārdomā vairākas potenciālas paralelitātes iespējas šajā problēmā.

%Laiksakritības identificēšana sākas ar problēmas dekompozīciju neatkarīgajās daļās, kuras
%var apstrādāt paralēli. Dekompozīcija var tikt veikta pēc uzdevumiem (ja eksistē uzdevumi,
%kas var tikt izpildīti paralēli) un pēc datiem (ja eksistē dati, ko var apstrādāt paralēli).
%Tad iegūtās daļas tiek analizētas, proti, tās tiek noteiktā veidā sagrupētas, tiek atrastas
%atkarības starp daļām un to grupām (piemēram, temporālas atkarības) un tiek atrisinātas
%problēmas ar datu koplietošanu. Pareiza uzdevuma dekompozīcija un analīze ir īpaši svarīga,
%jo no tās lielā mērā ir atkarīga rezultējoša projektējuma kvalitāte un kā secinājums~–-
%algoritma veiktspēja.

Algoritma struktūras etaps ir saistīts ar algoritmizācijas stratēģijas jeb algoritma
struktūras izvēli, kas vislabāk atbilstu iepriekšējā solī atrastajai potenciālajai paralelitātei.
Šajā līmenī projektētājs pārdomā, kā var izmantot vēl joprojām abstraktās paralelitātes iespējas.

%Algoritma struktūra var būt organizēta pēc uzdevumiem (uzdevumu paralelitātes un skaldi un
%valdi šabloni), pēc datiem (ģeometriskās dekompozīcijas un rekursīvu datu šabloni) un pēc
%datu plūsmas (konveijera un notikumu sakņotas dekompozīcijas šabloni).

Atbalsta struktūru etapā sākas algoritma pārveidošana programmā. Šis etaps ir vidusstadija
starp algoritma struktūras izvēli un realizācijas mehānismiem. Tas ir saistīts ar divām svarīgām
šablonu grupām: ar šabloniem, kas atspoguļo pieejas programmas loģiskajai strukturēšanai
(kā loģiski organizēt skaitļošanu?) un šabloniem, kas atspoguļo bieži izmantojamās koplietojamas
datu struktūras (kā efektīvi koplietot datus?).

%Pie programmas loģiskām struktūrām pieder tādas pieejas kā SPMD, vedējs-sekotājs (vedējs
%sadala uzdevumus starp sekotājiem, angliski: master/worker), ciklu paralēlisms (paņēmieni,
%kā paralelizēt ciklus) un sadalīt-un-apvienot (pēc nepieciešamības programma sadalās vairākās
%paralēlajās daļās, bet beigās atkal apvienojas vienā instancē, angliski: fork/join).

%Savukārt, pie datu struktūrām attiecas šādi šabloni: koplietojami dati (apraksta vispārīgus
%datu koplietošanas principus), koplietojama rinda (specializēts koplietojamu datu gadījums)
%un izkliedēts masīvs (apraksta, kā sadalīt lielu datu masīvu starp skaitļošanas elementiem).

Realizācijas mehānismu etaps ir saistīts ar to, kā augstākajos līmeņos atrastie risinājumi
un pieņemtie lēmumi tiek kartēti uz noteiktām programmēšanas valodām un/vai programmēšanas
vidēm. Šajā etapā algoritma izstrādātājam ir jāpielieto (varbūt, arī jāimplementē) konkrētus
procesu un pavedienu pārvaldības (piemēram, pavedienu un procesu izveidošanas un iznīcināšanas)
un sadarbošanās (piemēram, semaforu, barjeru vai ziņojumu nodošanas) mehānismus.

