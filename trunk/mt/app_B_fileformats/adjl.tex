\subsection{Blakusvirsotņu saraksts, *.adjl}
\begin{tabbing}
\hspace{3cm}\=\kill
Paplašinājums:\> *.adjl \\
Tips:\> Binārais
\end{tabbing}

\subsubsection*{Apraksts}
Šī vienkārša tipa datne tiek paredzēta grafa struktūras glabāšanai blakusvirsotņu saraksta
veidā. Datne ļauj glabāt multigrafus un norādīt loku svarus (32 bitu \emph{integer} tipa
skaitlis).

\subsubsection*{Struktūra}
\begin{tabbing}
\hspace{3cm}\=\hspace{3cm}\=\kill
\inbold{Nobīde, baiti}\> \inbold{Izmērs, baiti}\> \inbold{Paskaidrojums} \\
\emph{Datnes galvene:} \\
\(0\)\> 4\> Datnes paraksts: 0x34, 0x53, 0x23, 0x00 \\
\(4\)\> 4\> Grafa virsotņu skaits (\(s\))\\
\(8\)\> 4\(\cdot s\)\> Virsotņu apraksta nobīžu (4 baiti katrai nobīdei) masīvs \\

\\\emph{Virsotņu apraksti (\(i\)~-- kārtējās virsotnes apraksta nobīde):} \\
\(8+4\)\(\cdot s+i\)\> 4\> Izejošo loku skaits (\(n\))\\
\(12+4\cdot s+i\)\> 4\> Virsotnes numurs, kur ieiet pirmais loks \\
\(16+4\cdot s+i\)\> 4\> Pirmā loka svars \\
\ldots\> \ldots\> \ldots \\
\(12+4\cdot s+i+n-1\) \\
	\> 4\> Virsotnes numurs, kurā ieiet pēdējais loks \\
\(16+4\cdot s+i+n-1\) \\
	\> 4\> Pēdējā loka svars \\
\end{tabbing}

%\begin{newtable}{-0.2}{Kvalitatīvas analīzes rezultāti}{kval:res}
%	{8}{|p{0.5cm}|p{1.8cm}|p{1.8cm}|p{1.8cm}|p{1.8cm}|p{1.8cm}|p{1.8cm}|p{2.2cm}|}
%	{V & $R^1_{LP}$ & $R^1_{VC}$ & $R^2$ & $R^{total}_{LP}$ & $R^{total}_{VC}$ & $N(i)_{LP}$ & $N(i)_{VC}$}
%	15 & 6 & 15 & 3 & 7 & 14 & 0.25 & 0.0714286 \nextline
%	\multicolumn{8}{|l|}{$D_{LP}$ = 0.359375} \nextline%
%	\multicolumn{8}{|l|}{$D_{VC}$ = 0.0663265} \nextline
%\end{newtable}

