\subsection{MtException}
Klase \verb+MtException+ atspoguļo izņēmumu.
\begin{includes}
#include <QString>
#include "mt.h"
\end{includes}

\publicmethods
\begin{memblist}
	\membX{}{MtException}{mt\_ecode\_t code, const QString \&text}{nothrow}
	\membX{mt\_ecode\_t}{code}{}{const, nothrow}
	\membX{QString}{text}{}{const, nothrow}
\end{memblist}

\classdescription
Šī klase tiek izmantota klienta programmatūrā kā primārais kļūdu nodošanas un
par kļūdu apziņošanas mehānisms.

Komponente, kas grib paziņot par kļūdu, ,,met`` šo izņēmumu, norādot kļūdas kodu
un tekstu šīs klases konstruktorā:
\begin{code}
if ( someErrorCondition )
	throw MtException(MTE_ERROR_CODE, tr("Error text"));
\end{code}

Savukārt, komponente, kas grib apstrādāt kļūdas ziņojumu, ,,ķer`` doto izņēmumu.
Informāciju par kļūdu var saņemt no \verb+code+ un \verb+text+ metodēm:
\begin{code}[commandchars=^\{\}]
try ^{
	^ldots
^} catch ( MtException e ) ^{
	printMessage("Error code: %d; error text: %s",
	             e.code(), e.text());
^}
\end{code}

\methoddescription
\method{asd}%\verb|MtException(mt_ecode_t code, const QString &text)|}
Izveido izņēmumu ar kodu \emph{code} un tekstu \emph{text}.

