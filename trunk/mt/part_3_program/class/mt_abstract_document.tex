\subsection{MtAbstractDocument}
Klase \verb|MtAbstractDocument| atspoguļo abstraktu dokumentu: grafu, uzdevumu
vai risinājumus.
\begin{includes}
#include <QObject>
#include "MtException.hh"
\end{includes}

\publicmethods
\begin{memblist}
	\membX{}{MtAbstractDocument}{QObject *parent = 0}{explicit, nothrow}
	\membX{void}{open}{const QString \&fileName}{throw MtException}
	\membX{void}{save}{}{throw MtException}
	\memb{void}{close}{}
	\membX{QString}{caption}{}{const, nothrow}
	\membX{void}{setCaption}{const QString \&caption}{nothrow}
	\membX{QString}{fileName}{}{const, nothrow}
	\membX{void}{setFileName}{const QString \&fileName}{throw MtException}
	\membX{bool}{isModified}{}{const, nothrow}
\end{memblist}

\protectedmethods
\begin{memblist}
	\membX{void}{modify}{}{nothrow}
\end{memblist}

\signals
\begin{memblist}
	\memb{void}{opened}{}
	\memb{void}{saved}{}
	\memb{void}{closed}{}
	\memb{void}{modified}{}
\end{memblist}

\classdescription
Šī klase definē abstraktu dokumentu, ar ko var strādāt klienta programmatūra.
Šo klasi atvasina abstrakti grafu, uzdevumu un rezultātu dokumenti.

Abstraktam dokumentam tiek definētas šādas darbības: ielādēt dokumentu no datnes
(\verb|open()|), saglabāt to datnē (\verb|save()|) vai vienkārši aizvērt (\verb|close()|).
Jauna dokumenta izveidošanai ir jābūt implementētai atvasinātajās klasēs.

Abstraktam dokumentam piemīt šādas īpašības: nosaukums (\verb|caption|),
datnes vārds (\verb|fileName|) un modifikācijas flags (\verb|isModified|).

\methoddescription

