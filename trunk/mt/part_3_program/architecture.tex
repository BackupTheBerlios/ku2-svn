\hyphenation{sis-tē-mās}

\section{Programmatūras arhitektūra}
Programmatūras sistēmas arhitektūra balstās uz divu mēzglu koncepciju, līdzīgi kā
klient-servera arhitektūrās. Sauksīm vienu mēzglu par \emph{klientu} \engl{client},
bet otru~-- par \emph{izpildītāju} \engl{executor}. Šādas sistēms
konceptuālā struktūra tiek parādīta \reffig{prog-arch} attēlā. Kā ir redzams, tā
sastāv no četrām komponentēm:
\begin{numlist}
	\item klienta,
	\item grafu un uzdevumu,
	\item izpildītāja un
	\item rezultātu.
\end{numlist}
\numfig{Programmatūras sistēmas arhitektūra}{prog-arch}{1}{img/SW-Architecture}

\emph{Klienta komponente} atbild par grafiskās saskarnes izveidošanu, kas ļautu sistēmas
lietotājam
\begin{dotlist}
	\item izveidot un apskatīt grafus un uzdevumus un
	\item saņemt un atspoguļot uzdevumu izpildīšanas rezultātus no izpildītāja.
\end{dotlist}
Šī komponente ir atsevišķa programma, kas tiek izstrādāta C++ programmēšanas valodā,
izmantojot Qt bibliotēku (par izmantojamām programmēšanas valodām, bibliotēkām un
rīkiem tiks detalizētāk stastīts \ref{section:program:tools} sadaļā). Tā darba vide
ir klienta darba stacija ar jebkuru Qt atbalstītu operētājsistēmu, iekļaujot Microsoft
Windows un Linux.

\emph{Izpildītāja komponente} atbild par lietotāja izveidotu uzdevumu izpildi. Šī
komponente ir atsevišķa programma, kas tiek palaista uz klastera, proti, tā realizē
paralēlus uzdevumu risināšanas algoritmus. Programma tiek izstrādāta C99 standarta C valodā,
pielietojot MPI bibliotēku paralelitātes funkcijām. Platformspecifisku lietu neizmantošana
nodrošina šādas programmas labu parnēsajamību.

\emph{Grafu un uzdevumu} un \emph{rezultātu komponentes} atbild par grafu, uzdevumu
un rezultātu datu struktūru definēšanu, to glabāšanu, apstrādi un pārveidošanu.
Šīs komponentes tiek realizētas bibliotēku veidā. Tos izmanto gan klienta, gan
izpildītāja komponentēs. Abas bibliotēkas rakstā C99 standarta C valodā. Grafu
komponentē ietilpst arī dažādas palīgprogrammas, piemēram, ārējo datu formātu
importēšana vai noverģēšana sistēmas formātos.

