\section{Izstrādes rīki}\label{section:program:tools}
Šajā sadaļā tiek apkopota informācija par visiem programmatūras rīkiem, kas ir
izmantoti maģistra darba izstrādei.

\subsection{Subversion}
Interneta vietne: \emph{http://subversion.tigris.org/}

\noindent
Subversion ir bezmaksas atvērtā pirmteksta versiju kontroles sistēma. Tas
nozīme, ka Subversion pārvalda datnes un direktorijas, kā arī tajās
izdarītas izmaiņas laikā. Tas ļauj jebkurā laikā atgriezties pie vecākām datu
versijām vai apskatīties vēsturi, kā šie dati bija mainījušies. Šī iemesļa dēļ
versiju kontroles sistēmu var uzskatīt par kaut ko līdzīgu ,,laika mašīnai``
\cite{SVNBook}.

Subversion var strādāt tīklā, līdz ar ko, to paralēli var izmantot vairākos datoros.
Šajā gadījumā visas izmaiņas datos, kas tika veiktas vienā no datoriem, tiek
sinhronizētas ar visām citām datu kopijām. Šī Subversion (un citu versiju kontroles
sistēmu) īpatnība plaši tiek izmantota programmatūras projektos ar vairākiem
izstrādātājiem. Tā kā dati ir zem veriju kontroles, vienmēr var atcelt jaunākās
izmaiņas, ja tās noveda pie ta, ka projekts kļuva nedarbspējīgs.

Taču Subversion var izmantot arī viens izstrādātājs, ja viņam ir daudzas darba
stacijas, kur viņš izstrādā projektu. Es izmantoju Subversion tieši šiem nolūkiem.

\subsection{SCons}
Interneta vietne: \emph{http://www.scons.org/}

\noindent
SCons ir atvērtā pirmteksta programmatūras savākšanas rīks \engl{software
construction tool}. SCons ir uzlabots, platformneatkarīgs aizvietojums klasiskajai
\emph{Make} utilitai ar integrētu funcionalitāti līdzīgi \emph{autoconf/automake}
\cite{SCons}. Citiem vārdiem sakot, šī utilita pārvalda programmatūras
kompilēšanas procedūru, ievērojot komponenšu savstarpējas atkarības un citus
aspektus, kas ir vajadzīgas, lai no vairākiem pirmtekstiem, konfigurācijas datnēm
un resursiem savāktu programmu vai programmu pakotni kā vienu veselu.

Scons programmai ir šādas galvenas īpašības \cite{SCons}:
\begin{dotlist}
	\item Konfigurāciju datnes (skripti, kas kontrolē savākšanas procesu) ir
		Python valodā rakstītas programmas. Tas ļāuj efektīvi pielāgot rīku konkrētajam
		projektam.
	\item Automātikska komponenšu un moduļu atkarību analīze C un C++ valodām.
	\item Iebūvēts atbalsts C, C++, D, Java, Fortran, Yacc, Lex, Qt, SWIG un \LaTeX
		formātiem.
	\item Iebūvēts versiju kontroles sistēmu atbalsts.
	\item Microsoft Visual Studio atbalsts. Iespēja ģenerēt Visual Studio projektus.
	\item Paralēlas programmas savākšanas atbalsts. Dažādi moduļi var tikt kompilēti
		paralēli.
	\item Strādā Linux, citās POSIX sistēmās (iekļaujot AIX, *BSD, HP/UX, IRIX un
		Solaris), Windows NT, Mac OS X un OS/2.
\end{dotlist}

%\subsection{GNU Compiler Collection}
%\subsection{MinGW}
%\subsection{Microsoft Visual C}
\subsection{Qt}
Interneta vietne: \emph{http://qt.nokia.com/}

Qt (,,kju-te`` vai ,,kjut``) ir platformneatkarīgs programmatūras un lietotāja
saskarnes izstrādes ietvars. Izmantojot Qt ir iespējams izstrādāt kross-platformus
lietojumus ar grafisko lietotāju saskarni ar ,,uzraksti vienreiz, kompilē visur``
pieeju. Tas nozīme, ka vienreiz uzrakstīta programma var tikt pārkompilēta un palaista
Windows, Linux, Mac OS X un citās operētājsistēmās \cite{QtNokia}.

Tādas kompānijas un organizācijas kā Adobe®, Boeing®, Google®, IBM®, Motorola®,
NASA un Skype® izmanto Qt C++ ietvaru savos projektos.

Qt ir pilnīgi objektorientēts. Qt sastāvā ir liels skaits lietotāju saskarnes
elementu jeb vidžetu \engl{widget}, ko var izmantot sarežģītu saskarņu veidošanai.

%Qt’s cross-platform GUI applications
%can support all the user interface functionality required by modern applications, such as menus,
%context menus, drag and drop, and dockable toolbars. Desktop integration features
%provided by Qt can be used to extend applications into the surrounding desktop environment,
%taking advantage of some of the services provided on each platform.%

%Qt has excellent support for multimedia and 3D graphics. Qt is the de facto standard
%GUI framework for platform-independent OpenGL® programming. Qt’s painting system offers
%high quality rendering across all supported platforms. A sophisticated canvas framework
%enables developers to create interactive graphical applications that take advantage of Qt’s
%advanced painting features.

%Qt makes it possible to create platform-independent database applications using standard data-
%bases. Qt includes native drivers for Oracle®, Microsoft® SQL Server, Sybase® Adap-
%tive Server, IBM DB2®, PostgreSQLTM, MySQL®, Borland® Interbase, SQLite, and ODBC-compliant
%databases. Qt includes database-specific widgets, and any built-in or custom widget can be made
%data-aware.

%Qt programs have native look and feel on all supported platforms using Qt’s styles and themes
%support. From a single source tree, recompilation is all that is required to produce
%applications for Windows® XP®and Windows VistaTM, Mac OS X®, Linux®, SolarisTM, HP-UXTM,
%and many other versions of Unix® with X11TM. Qt’s qmake build tool produces makefiles or .dsp
%files appropriate to the target platform.

%Since Qt’s architecture takes advantage of the underlying platform, many customers use Qt for
%single-platform development on Windows, Mac OS X, and Unix because they prefer Qt’s approach.
%Qt includes support for important platform-specific features, such as ActiveX® on Windows, and
%MotifTM on Unix.

%Qt uses UnicodeTM throughout and has considerable support for internationalization.
%Qt includes Qt Linguist and other tools to support translators. Applications can easily use and mix
%text in Arabic, Chinese, English, Hebrew, Japanese, Russian, and other languages supported by
%Unicode.

%Qt includes a variety of domain-specific classes. For example, Qt has an XML module that
%includes SAX and DOM classes for reading and manipulating data stored in XML-based formats.
%Objects can be stored in memory using Qt’s STL-compatible collection classes, and
%handled using styles of iterators used in Java® and the C++ Standard Template Library (STL).
%Local and remote file handling using standard protocols are provided by Qt’s input/output and
%networking classes.

%Qt applications can have their functionality extended by plugins and dynamic libraries.
%Plugins provide additional codecs, database drivers, image formats, styles, and widgets. Plugins
%and libraries can be sold as products in their own right.

%The QtScript module enables applications to be scripted with Qt Script, an ECMAScript-
%based language related to JavaScript. This technology allows developers to give users restricted
%access to parts of their applications for scripting purposes.

%Qt is a mature C++ framework that is widely used around the world. In addition to Qt’s many
%commercial uses, the Open Source edition of Qt is the foundation of KDE, the Linux desktop en-
%vironment. Qt makes application development a pleasure, with its cross-platform build system,
%visual form design, and elegant API.

%\subsection{Doxygen}
%\subsection{XeLaTeX}

