\subsection{gentsp}
\emph{gentsp} --- nejaušā veidā ģenerē komivojažiera problēmu (TSP).
\begin{shell}[commandchars=\\\{\}]
% gentsp --type \underline{TSP tips} -n \underline{virsotņu skaits} -o \underline{izejas datne}
\end{shell}

\subsubsection*{Apraksts}
Programma \emph{gentsp} izmanto nejaušo skaitļu generatoru, lai izveidotu noteikta
tipa komivojažiera problēmas instanci ar noteiktiem parametriem. Pēc noklusēšanas
rezultāts tiek izvadīts standarta izejā.

\subsubsection*{Parametri}
\DefineShortVerb{\@}
\begin{desclist}
	\item @--about@ \\
		izvadīt uz ekrāna informāciju par programmatūru: versiju, autoru, utt;
	\item @--help@ \\
		izvadīt uz ekrāna palīdzību programmas izmantošanā: programmas palaišana,
		parametru apraksts, utt;
	\item @--type [:TSP type:]@ \\
		izvelēties generējamās komivojažiera problēmas tipu, iespējamie	tipi ir šādi:
		\begin{desclist}
			\item @geom@ --- ģeomētriskā TSP (vērtība pēc noklusēšanas);
		\end{desclist}
	\item @--geom@ \\ generēt ģeomētrisko TSP (@--type geom@ alternatīva);
	\item @--nodes [0-9]+@ vai @-n [0-9]+@ \\
		noteikt virsotņu skaitu (pēc noklusēšanas -- 15);
	\item @--output [:filename:]@ vai @-o [:filename:]@ \\
		noteikt izejas datni (pēc noklusēšanas -- standarta izeja \underline{stdout}).
\end{desclist}
\UndefineShortVerb{\@}

\subsubsection*{Izmantošanas piemēri}

