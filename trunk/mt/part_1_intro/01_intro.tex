\hyphenation{sis-tē-mās}

%\section{Ievads}
Maģistra darba mērķis ir izpētīt:
\begin{dotlist}
	\item ievadu paralēlajā skaitļošanā un paralēlo datoru klasifikācijā,
	\item secīgu algoritmu pārveidošanu paralēlajos algoritmos, galvenos aspektus un pieejas,
	\item MPI tehnoloģiju,
	\item paralēlās programmēšanas problēmas un grūtības,
	\item grafu un paralelitātes sakarības,
	\item esošos paralēlos algoritmus grafu apstrādei un analīzei.
\end{dotlist}

Darbs loģiski tiek sadalīts divās daļās. \emph{Pirmā daļa} ir veltīta paralelitātei kā
tādai: definīcijai, pielietojumiem, paralēlo sistēmu arhitektūrām un topoloģijām,
kopējām stratēģijām paralēlo algoritmu izstrādei un paralēlajai programmēšanai
(MPI, atkļūdošanai, efektivitātes novērtēšanai).

\emph{Otrā daļa} ir veltīta grafiem kā vienai no uzdevumu klasēm, kas var tikt
risināta paralēlajās sistēmās. Parāda paralēlās skaitļošanas pielietošanu uzdevumos,
kas ir saistīti ar lielu grafu apstrādi. Daudzi no šiem uzdevumiem ir ar augstu
algoritmisku sarežģītību, proti, nav atrisināmi lielam datu apjomam pieņemamā
laikā uz viena procesora. Taču, lai pielietotu paralēlo skaitļošanu grafu uzdevumu
risināšanai, vispirms grafu algoritms ir ,,jāpārceļ`` no secīga uz paralēlu. Piemēram,
daži no šī procesa pamatuzdevumiem ir grafa un apakšuzdevumu sadalīšana starp
skaitļošanas elementiem un efektīvas komunikācijas nodrošināšana starp tiem. Šajā
daļā tiek apskatīts, kādas problēmas ir jārisina grafu algoritmu paralelizēšanas
laikā sistēmās ar izkliedētu atmiņu, kā arī, ar ko un ar kādām grafu īpašībām un
īpatnībām tās ir saistītas.

